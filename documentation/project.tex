\documentclass[11pt,a4paper,oneside, openright]{article}
\usepackage{graphicx}
\usepackage[british]{babel}
\usepackage[utf8]{inputenc}
\usepackage{mathtools}
\usepackage{setspace}
\usepackage{verbatim}
\usepackage{tikz}
\usetikzlibrary{trees}
\usepackage{url}
\usepackage{amsmath}

\begin{document}
{\setstretch{1.0}
  \begin{titlepage}
  	\centering
  	\includegraphics[width=6cm]{images/unipi.eps}\par
  	\vspace{1.5cm}
  	{\huge\textsc{Performance evaluation of a CRAN system}\par}
  	\vspace{2cm}
  	Gerardo \textsc{Alvaro}\par
  	Francesco \textsc{Barbarulo}\par
    Francesco \textsc{Fornaini}

  	\vfill

    % Bottom of the page
  	{\large 2018-2019\par}
  \end{titlepage}
}


\tableofcontents

\newpage

\section{Introduction}
\label{sec:introduciton}

The system examined is a simplified version of an architecture presented for future cellular networks, called Cloud-RAN (CRAN).

\subsection{Description of the system}
 At the center of this system there is a central processing unit (BBU), which is responsible for forwarding the packets received from an Application Server (AS) to one of the N remote radios (RRH) connected to it. Each RRH serves a single cell, and each packet generated by the AS has one of these cells as a destination, taken uniformly from those available. Each data packet has a size $s$ and a new one is generated every $t$ seconds. The BBU has an interface to each of the RRHs and communicates with only one of them at a time, at a speed of X bytes/s. If the BBU interface with the RRHs is busy, the data packets are queued and served using the FIFO policy.

When the BBU receives a packet from the AS, it can operate in two different ways:
\begin{itemize}
	\item[A)]It retransmits directly the packet to the RRH which serves the destination cell N;
	\item[B)]The BBU compress the packet, reducing its size by C\% and retransmitting it to the proper RRH. Once arrived at the RRH, the packet is decompressed. Such operation takes S seconds, where S is given by $ S = C \cdot 50ms $. Only one packet can be decompressed at a time. If the decompressing process is busy, the incoming data packets are queued and served using a FIFO policy.
\end{itemize}

Packet size and interarrivals are random variables described as:
\begin{itemize}
	\item exponential distribution of $t$;
	\item exponential and lognormal distribution of $s$.
\end{itemize}

\subsection{Objectives and performance indexes}
The objective of the study is to determine if and under what conditions it is better to perform packet compression (and in which percentage) or not.
For a correct evaluation of the system will be taken as reference the mean end-to-end delay of packets.

\section{Model}
Before introducing the model, shown in Figure~\ref{fig:model}, we make some semplifications which do not affect the final results.

\begin{figure}[h]
	\centering
	\includegraphics[width=0.9\textwidth]{images/model}
	\caption{Queuing network model}
	\label{fig:model}
\end{figure}

\subsection{Assumptions}
First of all we make some assuptions:
\begin{itemize}
    \item Propagation times ($ T_{prop} $) are negligible;
    \item BBU switching time (time needed to change the output gate) is negligible;
    \item In case B, the compression time on BBU is negligible;
    \item Packets are not corrupted;
    \item No packet loss at the buffers (infinite buffers).
\end{itemize}

With these assumptions, we can compute the end-to-end delay as following:

$$ T_{delay} =  T_{waiting\_bbu} + T_{transmission} + T_{waiting\_rrh} + T_{decompression} $$

where 

$$ T_{transmission} = \frac{1}{\mu_{bbu}} = \frac{s}{X} $$

and

$$ T_{decompression} = \frac{1}{\mu_{rrh}} = C \cdot 50ms $$

\subsection{Validation}
The simulator has been validated through a comparison with a queueing theory model of the system.

The system has been modelled as an Open Jackson Network because all the hypotheses are verified since external arrivals are poissonian and routing probabilities are state-independent because they are uniform as specified in the requirements.

\subsubsection{Stability conditions}
In order to compute the stabilty conditions we have to distinguish between the two cases.

In case A we have to respect only one condition regarding the BBU. Indeed, on the RRHs we will never have queues because when the packet arrives, it is immediately consumed.
Hence, the stability condition is the following:

$$ \lambda_{bbu} < \mu_{bbu}, \quad \lambda_{bbu} = \frac{1}{t}, \quad \mu_{bbu} = \frac{X}{s}$$

\begin{equation} \label{eq:rho-bbu}
\rho_{bbu} = \frac{\lambda_{bbu}}{\mu_{bbu}} = \frac{s}{t \cdot X} < 1
\end{equation}

Note that $s$ represents the number of bytes that the BBU has to transmit. From now on we will consider:
$$s = s\cdot(1-\frac{C}{100})$$

In case B we have also to take in account the stability condition on the RRHs because now the RRH service time is not null and depends on the compression percentage. 

We have to guarantee that the arrival rate of the RRHs is poissonian. If we consider just one RRH, the model can be seen as a tandem QN which respects Burke's theorem. Hence, the arrivals at the RRH are a Poisson process with rate $ \lambda_{bbu} $ regardless of the BBU service rate, that, in our simulations, will be both exponential and lognormal. 
Obviously, this reasoning can be extended in a system with more than one RRH, where the arrival rate will depend on the number of remote radios as follows:

$$ \lambda_{rrh} = \lambda_{bbu} \cdot \pi = \frac{1}{t \cdot N} $$

$$ \mu_{rrh} = \frac{1}{C \cdot k}, \quad k = 50ms $$

\begin{equation} \label{eq:rho-rrh}
\rho_{rrh} = \frac{\lambda_{rrh}}{\mu_{rrh}} = \frac{C \cdot k}{t \cdot N} < 1
\end{equation}
% questo è ok in condizione di stabilità, ma queste SONO le condizioni di stabilità

Hence, using the result obtained in \eqref{eq:rho-bbu}, the final stability condition for case B is:

$$ \begin{cases} \rho_{bbu} = \frac{s}{t \cdot X} < 1 \\ \\ \rho_{rrh} = \frac{C \cdot k}{t \cdot N} < 1 \end{cases} $$

with steady state probability, either on BBU and RRHs, equal to:

$$ p_{n} = \rho^n \cdot (1 - \rho) $$

Note that the steady state probability on each RRH and on the BBU is equal to an M/M/1.

\subsection{Statistics for validation}
In order to validate our model we have used the subsequent perfomance indexes taken from the queueing theory:

$$ E[N_{bbu}] = \frac{\rho_{bbu}}{1 - \rho_{bbu}} = \frac{s}{X \cdot t - s}$$

$$ E[R_{bbu}] = \frac{E[N_{bbu}]}{\lambda_{bbu}} $$

$$ E[N_{rrh}] = \frac{\rho_{rrh}}{1 - \rho_{rrh}} = \frac{C \cdot k}{t \cdot N - C \cdot k}$$

$$ E[R_{rrh}] = \frac{E[N_{rrh}]}{\lambda_{rrh}} $$
%?
The successful comparison between the previous equations and the simulator results assure us that we are working on a correct model for our system.
%?

\section{Simulator implementation}
The simulator consists of four modules as shown in Figure~\ref{fig:simulator}:
\begin{itemize}
  \item \texttt{As}: it creates/produces a packet flow according to the interarrival time and send them to the BBU;
  \item \texttt{Bbu}: it forwards the packets received from the AS to the RRH, compressed if it has to;
  \item \texttt{Rrh}: it decompresses the packet, if it was compressed, and send it to the collector for the delay statistics;
  \item \texttt{Collector}: it handles of the delay statistics.
\end{itemize}

\begin{figure}[h]
    \centering
    \includegraphics[width=0.9\textwidth]{images/simulator}
    \caption{Simulator architecture}
    \label{fig:simulator}
\end{figure}

\subsection{Application Server (AS)}
The As module has to perform cyclically the following operations:
\begin{itemize}
    \item[1.] it creates a new packet with the \texttt{id}, the packet size \texttt{s} taken either from an exponential distribution or a lognormal one, the destination taken from an uniform distribution, the creation time \texttt{created\_at};
    \item[2.] it sends the packet to the \textit{Bbu};
    \item[3.] it waits according to the interarrival time \texttt{t} taken from an exponential distribution.
\end{itemize}

\subsection{Baseband Unit (BBU)}
When the Bbu module receives a packet from the AS it has to do the following actions:
\begin{itemize}
    \item[1.] if it is idle, it processes the packet immediately, otherwise the packet will be queued and served using a FIFO policy;
    \item[2.] it processes the packet deciding whether the packet must be compressed or not and transmitting it to the proper Rrh;
    \item[3.] if there are any other packets in the queue, the first of them is pulled off from the buffer and served, otherwise it waits for the next packet.
\end{itemize}

\subsection{Remote radio (RRH)}
When one of the N Rrhs receives a packet from the Bbu, it acts as follow:
\begin{itemize}
    \item[1.] if it is idle, it processes the packet immediately, otherwise the packet will be queued and served using a FIFO policy;
    \item[2.] it processes the packet deciding whether the packet must be decompressed or not and transmitting it to the Collector;
    \item[3.] if there are any other packets in the queue, the first of them is pulled off from the buffer and served, otherwise it waits for the next packet.
\end{itemize}


\subsection{Collector}
The Collector module (virtually) collects all the packets coming from the RRHs and deals with the end-to-end delay statistics.

\section{Verification}
The simulator has been verified in order to check for memory leaks and bugs.
For the first ones we have used Valgrind tool, whereas for bugs we have done some simulations with known results.

After that we have set proper values, according to some simulation results, for both \texttt{warmup-time} and \texttt{simulation-time-limit}.

We have estimated the \texttt{warmup-time} applying the sliding window average (windowSize = ???) to the mean end-to-end delay values obtained from 20 independent repetitions of the highest value of $ \rho_{bbu} $ (0.9).
//figure

In Figure~\ref{fig:warm-up-study} it is shown that a good value for \texttt{warmup-time} is around ??? seconds, where the steady state is reached. 

\begin{figure}[h]
    \centering
    \includegraphics[width=0.9\textwidth]{images/warm-up}
    \caption{Warm-up study}
    \label{fig:warm-up-study}
\end{figure}

The value for the \texttt{simulation-time-limit} has been chosen in order to have enough results for statistics.

\section{Experiments}
We will analyze separately the two cases described in Section~\ref{sec:introduction} either with exponential and lognormal distribution, the latter only for the packet size.
We will take in account scenarios with at least 2 remote radios, because a system with only one remote radio is not interesting to analyze. Futhermore, from the stability condition of the \textit{Rrh} computed in \eqref{eq:rho-rrh}, we can not use less than 2 remote radios in order to analyze the case with compression between 10\% and 50\%, that, instead, it is our scope.

\subsection{Case A - Transmission without compression}
We have done 30 different repetitions for each single scenario. In this case the end-to-end delay could depend on two factors:
\begin{itemize}
	\item \texttt{X} chosen such as $ \rho_{bbu} $, computed as in \eqref{eq:rho-bbu}, will assume values from 0.1 to 0.9 increased by 0.1;
	\item \texttt{N} with prefixed values of 2, 5, 10.
\end{itemize}
\subsubsection{Exponential}
Since the exponential distribution does not provide a variance control, the packet size \texttt{s} extracted by the AS could assume both small and large values.
In Figure~\ref{fig:exp-a} the simulation results are plotted.
%figura
As we expected the end-to-end delay does not depend on the number of remote radios \texttt{N}, because on the RRH the service time is null, so there will be no queues and the packets will be forwarded to the cells immediately.
Hence, the only factor that allows to minimize the end-to-end delay is the BBU transmission speed \texttt{X}.

The end-to-end delay is computed as the sum of the mean BBU waiting time (shown in Figure~\ref{fig:waiting}) and the BBU service time.

%\begin{align}
%E[R] &= E[R_{bbu}] \notag \\
%&= E[W_{bbu}] + \frac{1}{\mu_{bbu}} \notag \\
%\end{align}
\subsubsection{Lognormal}
In the lognormal distribution we have used, the probability to have packet size values greater than the mean value is higher with respect to the exponential distribution. For this reason the end-to-end (shown in Figure~\ref{fig:log-a}) delay tends to be higher than the previous analysis as $\rho$ increases.

Of course, also in this case, the only meaningful scalable factor is the BBU transmission speed.

































\end{document}